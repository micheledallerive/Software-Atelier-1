\documentclass[a4paper,12pt]{article}
\usepackage[utf8]{inputenc}

\newcommand{\<}{\textless}
\renewcommand{\>}{\textgreater}

\begin{document}

\begin{center}
 \huge{\textbf{Git guide}}\\[.5em]
 \Large{Project 2 - How to use our git repository}
\end{center}
\vspace{3em}
\tableofcontents

\newpage

\section{Introduction}
Git is a versioning system that can handle all the files of a project. It helps with tracking all the version and the modification by all the team members. Keep in mind that git is a very powerful tool and you should always pay attention when you use it.\\[.5em]
If you have \textbf{\underline{ANY}} doubt about GIT just ask the GIT Leader Riccardo (+39 3518878327) or the team leaders (Michele: +39 3487521909, Sofia: +39 3735325273).

\section{How to clone the repo}
In order to get the content of our git repository you need to clone it:
\begin{center}
 \small{\texttt{git clone <username>@atelier.inf.usi.ch:/home/lambrr/project\_sa1.git}}\\[.75em]
 \textless username\textgreater \enspace is your short username (e.g. dallem)
\end{center}

\section{File creation guidelines}
In order to create an organized folder you need to follow this guidelines:
\begin{itemize}
 \item \textbf{Biography images}: The images of each Award winner's biography (e.g. Berners-Lee's biography) must follow the format:\\ 
 \texttt{./src/images/decades/\<decade of the award\>/\<year of the\\winner's win\>.(png or jpg)}\\[.33em]
 e.g. ./src/images/decades/2010/2016.png (picture of Berners-Lee)
 
 \item \textbf{Award images}: The images of each Award winning reason (e.g. the invention of the World Wide Web) must follow the format:\\
 \texttt{./src/images/decades/\<decade of the award\>/\\award\_\<year of the award\>.(png or jpg)}\\[.33em] 
 e.g. ./src/images/decades/2010/award\_2016.png (image of the WWW)
 
 
\end{itemize}


\section{Submit the code changes}
Once you downloaded the directory you can start addeding pages, images and editing the files. 
\textbf{Remember} that you need to follow the CSS guidelines and the following 


\end{document}
